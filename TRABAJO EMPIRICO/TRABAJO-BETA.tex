% Options for packages loaded elsewhere
\PassOptionsToPackage{unicode}{hyperref}
\PassOptionsToPackage{hyphens}{url}
%
\documentclass[
]{article}
\usepackage{lmodern}
\usepackage{amssymb,amsmath}
\usepackage{ifxetex,ifluatex}
\ifnum 0\ifxetex 1\fi\ifluatex 1\fi=0 % if pdftex
  \usepackage[T1]{fontenc}
  \usepackage[utf8]{inputenc}
  \usepackage{textcomp} % provide euro and other symbols
\else % if luatex or xetex
  \usepackage{unicode-math}
  \defaultfontfeatures{Scale=MatchLowercase}
  \defaultfontfeatures[\rmfamily]{Ligatures=TeX,Scale=1}
\fi
% Use upquote if available, for straight quotes in verbatim environments
\IfFileExists{upquote.sty}{\usepackage{upquote}}{}
\IfFileExists{microtype.sty}{% use microtype if available
  \usepackage[]{microtype}
  \UseMicrotypeSet[protrusion]{basicmath} % disable protrusion for tt fonts
}{}
\makeatletter
\@ifundefined{KOMAClassName}{% if non-KOMA class
  \IfFileExists{parskip.sty}{%
    \usepackage{parskip}
  }{% else
    \setlength{\parindent}{0pt}
    \setlength{\parskip}{6pt plus 2pt minus 1pt}}
}{% if KOMA class
  \KOMAoptions{parskip=half}}
\makeatother
\usepackage{xcolor}
\IfFileExists{xurl.sty}{\usepackage{xurl}}{} % add URL line breaks if available
\IfFileExists{bookmark.sty}{\usepackage{bookmark}}{\usepackage{hyperref}}
\hypersetup{
  pdftitle={DISTRIBUCIONBETA},
  pdfauthor={Yo},
  hidelinks,
  pdfcreator={LaTeX via pandoc}}
\urlstyle{same} % disable monospaced font for URLs
\usepackage[margin=1in]{geometry}
\usepackage{color}
\usepackage{fancyvrb}
\newcommand{\VerbBar}{|}
\newcommand{\VERB}{\Verb[commandchars=\\\{\}]}
\DefineVerbatimEnvironment{Highlighting}{Verbatim}{commandchars=\\\{\}}
% Add ',fontsize=\small' for more characters per line
\usepackage{framed}
\definecolor{shadecolor}{RGB}{248,248,248}
\newenvironment{Shaded}{\begin{snugshade}}{\end{snugshade}}
\newcommand{\AlertTok}[1]{\textcolor[rgb]{0.94,0.16,0.16}{#1}}
\newcommand{\AnnotationTok}[1]{\textcolor[rgb]{0.56,0.35,0.01}{\textbf{\textit{#1}}}}
\newcommand{\AttributeTok}[1]{\textcolor[rgb]{0.77,0.63,0.00}{#1}}
\newcommand{\BaseNTok}[1]{\textcolor[rgb]{0.00,0.00,0.81}{#1}}
\newcommand{\BuiltInTok}[1]{#1}
\newcommand{\CharTok}[1]{\textcolor[rgb]{0.31,0.60,0.02}{#1}}
\newcommand{\CommentTok}[1]{\textcolor[rgb]{0.56,0.35,0.01}{\textit{#1}}}
\newcommand{\CommentVarTok}[1]{\textcolor[rgb]{0.56,0.35,0.01}{\textbf{\textit{#1}}}}
\newcommand{\ConstantTok}[1]{\textcolor[rgb]{0.00,0.00,0.00}{#1}}
\newcommand{\ControlFlowTok}[1]{\textcolor[rgb]{0.13,0.29,0.53}{\textbf{#1}}}
\newcommand{\DataTypeTok}[1]{\textcolor[rgb]{0.13,0.29,0.53}{#1}}
\newcommand{\DecValTok}[1]{\textcolor[rgb]{0.00,0.00,0.81}{#1}}
\newcommand{\DocumentationTok}[1]{\textcolor[rgb]{0.56,0.35,0.01}{\textbf{\textit{#1}}}}
\newcommand{\ErrorTok}[1]{\textcolor[rgb]{0.64,0.00,0.00}{\textbf{#1}}}
\newcommand{\ExtensionTok}[1]{#1}
\newcommand{\FloatTok}[1]{\textcolor[rgb]{0.00,0.00,0.81}{#1}}
\newcommand{\FunctionTok}[1]{\textcolor[rgb]{0.00,0.00,0.00}{#1}}
\newcommand{\ImportTok}[1]{#1}
\newcommand{\InformationTok}[1]{\textcolor[rgb]{0.56,0.35,0.01}{\textbf{\textit{#1}}}}
\newcommand{\KeywordTok}[1]{\textcolor[rgb]{0.13,0.29,0.53}{\textbf{#1}}}
\newcommand{\NormalTok}[1]{#1}
\newcommand{\OperatorTok}[1]{\textcolor[rgb]{0.81,0.36,0.00}{\textbf{#1}}}
\newcommand{\OtherTok}[1]{\textcolor[rgb]{0.56,0.35,0.01}{#1}}
\newcommand{\PreprocessorTok}[1]{\textcolor[rgb]{0.56,0.35,0.01}{\textit{#1}}}
\newcommand{\RegionMarkerTok}[1]{#1}
\newcommand{\SpecialCharTok}[1]{\textcolor[rgb]{0.00,0.00,0.00}{#1}}
\newcommand{\SpecialStringTok}[1]{\textcolor[rgb]{0.31,0.60,0.02}{#1}}
\newcommand{\StringTok}[1]{\textcolor[rgb]{0.31,0.60,0.02}{#1}}
\newcommand{\VariableTok}[1]{\textcolor[rgb]{0.00,0.00,0.00}{#1}}
\newcommand{\VerbatimStringTok}[1]{\textcolor[rgb]{0.31,0.60,0.02}{#1}}
\newcommand{\WarningTok}[1]{\textcolor[rgb]{0.56,0.35,0.01}{\textbf{\textit{#1}}}}
\usepackage{graphicx,grffile}
\makeatletter
\def\maxwidth{\ifdim\Gin@nat@width>\linewidth\linewidth\else\Gin@nat@width\fi}
\def\maxheight{\ifdim\Gin@nat@height>\textheight\textheight\else\Gin@nat@height\fi}
\makeatother
% Scale images if necessary, so that they will not overflow the page
% margins by default, and it is still possible to overwrite the defaults
% using explicit options in \includegraphics[width, height, ...]{}
\setkeys{Gin}{width=\maxwidth,height=\maxheight,keepaspectratio}
% Set default figure placement to htbp
\makeatletter
\def\fps@figure{htbp}
\makeatother
\setlength{\emergencystretch}{3em} % prevent overfull lines
\providecommand{\tightlist}{%
  \setlength{\itemsep}{0pt}\setlength{\parskip}{0pt}}
\setcounter{secnumdepth}{-\maxdimen} % remove section numbering
\usepackage{booktabs}
\usepackage{longtable}
\usepackage{array}
\usepackage{multirow}
\usepackage{wrapfig}
\usepackage{float}
\usepackage{colortbl}
\usepackage{pdflscape}
\usepackage{tabu}
\usepackage{threeparttable}
\usepackage{threeparttablex}
\usepackage[normalem]{ulem}
\usepackage{makecell}
\usepackage{amsmath}
\usepackage{caption}

\title{DISTRIBUCIONBETA}
\author{Yo}
\date{10/2/2020}

\begin{document}
\maketitle

\begin{Shaded}
\begin{Highlighting}[]
\KeywordTok{library}\NormalTok{(tidyverse)}
\end{Highlighting}
\end{Shaded}

\begin{verbatim}
## -- Attaching packages --------------------------------------------------------------------------------- tidyverse 1.3.0 --
\end{verbatim}

\begin{verbatim}
## v ggplot2 3.3.2     v purrr   0.3.4
## v tibble  3.0.1     v dplyr   1.0.0
## v tidyr   1.1.0     v stringr 1.4.0
## v readr   1.3.1     v forcats 0.5.0
\end{verbatim}

\begin{verbatim}
## -- Conflicts ------------------------------------------------------------------------------------ tidyverse_conflicts() --
## x dplyr::filter() masks stats::filter()
## x dplyr::lag()    masks stats::lag()
\end{verbatim}

\begin{Shaded}
\begin{Highlighting}[]
\KeywordTok{library}\NormalTok{(e1071)}
\end{Highlighting}
\end{Shaded}

\begin{verbatim}
## Warning: package 'e1071' was built under R version 4.0.2
\end{verbatim}

\begin{Shaded}
\begin{Highlighting}[]
\KeywordTok{library}\NormalTok{(moments)}
\end{Highlighting}
\end{Shaded}

\begin{verbatim}
## 
## Attaching package: 'moments'
\end{verbatim}

\begin{verbatim}
## The following objects are masked from 'package:e1071':
## 
##     kurtosis, moment, skewness
\end{verbatim}

\begin{Shaded}
\begin{Highlighting}[]
\KeywordTok{library}\NormalTok{(gt)}
\end{Highlighting}
\end{Shaded}

\begin{verbatim}
## Warning: package 'gt' was built under R version 4.0.2
\end{verbatim}

\begin{Shaded}
\begin{Highlighting}[]
\KeywordTok{library}\NormalTok{(ggpubr)}
\end{Highlighting}
\end{Shaded}

\begin{verbatim}
## Warning: package 'ggpubr' was built under R version 4.0.2
\end{verbatim}

\begin{Shaded}
\begin{Highlighting}[]
\KeywordTok{library}\NormalTok{(ggplotify)}
\end{Highlighting}
\end{Shaded}

\begin{verbatim}
## Warning: package 'ggplotify' was built under R version 4.0.2
\end{verbatim}

\begin{Shaded}
\begin{Highlighting}[]
\KeywordTok{library}\NormalTok{(grid)}
\CommentTok{# 1. Creamos las muestras, evaluamos el estadístico en ellas, así como la media muestral}

\NormalTok{muestreo <-}\StringTok{ }\ControlFlowTok{function}\NormalTok{(alpha, beta)\{}
\NormalTok{        lista <-}\StringTok{ }\KeywordTok{list}\NormalTok{()}
\NormalTok{        tabla_muestras <-}\StringTok{ }\KeywordTok{matrix}\NormalTok{(}\DataTypeTok{ncol =} \DecValTok{40}\NormalTok{, }\DataTypeTok{nrow =} \DecValTok{10}\NormalTok{) }\OperatorTok\StringTok{ }\KeywordTok{as.data.frame}\NormalTok{()}
\NormalTok{        lista[[}\DecValTok{6}\NormalTok{]] <-}\StringTok{ }\KeywordTok{matrix}\NormalTok{(}\DataTypeTok{ncol =} \DecValTok{40}\NormalTok{, }\DataTypeTok{nrow =} \DecValTok{1}\NormalTok{) }\OperatorTok\StringTok{ }\KeywordTok{as.data.frame}\NormalTok{()}
\NormalTok{        media_poblacional <-}\StringTok{ }\NormalTok{alpha }\OperatorTok{/}\StringTok{ }\NormalTok{(alpha }\OperatorTok{+}\StringTok{ }\NormalTok{beta)}
\NormalTok{        varianza_poblacional <-}\StringTok{ }\NormalTok{alpha }\OperatorTok{*}\StringTok{ }\NormalTok{beta }\OperatorTok{/}\StringTok{ }\NormalTok{(((alpha }\OperatorTok{+}\StringTok{ }\NormalTok{beta) }\OperatorTok{^}\StringTok{ }\DecValTok{2}\NormalTok{ ) }\OperatorTok{*}\StringTok{ }\NormalTok{(alpha }\OperatorTok{+}\StringTok{ }\NormalTok{beta }\OperatorTok{+}\StringTok{ }\DecValTok{1}\NormalTok{))}
        \ControlFlowTok{for}\NormalTok{(j }\ControlFlowTok{in} \DecValTok{1}\OperatorTok{:}\DecValTok{40}\NormalTok{)\{}
                \KeywordTok{set.seed}\NormalTok{(j)}
\NormalTok{                tabla_muestras[, j] <-}\StringTok{ }\KeywordTok{rbeta}\NormalTok{(}\DecValTok{10}\NormalTok{, }\DataTypeTok{shape1 =}\NormalTok{ alpha, }\DataTypeTok{shape2 =}\NormalTok{ beta)}
                \KeywordTok{colnames}\NormalTok{(tabla_muestras)[j] <-}\StringTok{ }\KeywordTok{as.numeric}\NormalTok{(}\KeywordTok{gsub}\NormalTok{(}\StringTok{"V"}\NormalTok{, }\StringTok{""}\NormalTok{, }\KeywordTok{colnames}\NormalTok{(tabla_muestras)[j]))}
                \KeywordTok{colnames}\NormalTok{(lista[[}\DecValTok{6}\NormalTok{]])[j] <-}\StringTok{  }\KeywordTok{as.numeric}\NormalTok{(}\KeywordTok{gsub}\NormalTok{(}\StringTok{"V"}\NormalTok{, }\StringTok{""}\NormalTok{, }\KeywordTok{colnames}\NormalTok{(tabla_muestras)[j]))}
\NormalTok{                lista[[}\DecValTok{6}\NormalTok{]][, j] <-}\StringTok{ }\FloatTok{0.5}\OperatorTok{*}\NormalTok{(}\KeywordTok{quantile}\NormalTok{(tabla_muestras[, j], }\DataTypeTok{probs =} \KeywordTok{c}\NormalTok{(}\FloatTok{0.6}\NormalTok{)) }\OperatorTok{+}\StringTok{ }\KeywordTok{quantile}\NormalTok{(tabla_muestras[, j], }\DataTypeTok{probs =} \KeywordTok{c}\NormalTok{(}\FloatTok{0.4}\NormalTok{)))}
\NormalTok{        \}}
\NormalTok{        lista[[}\DecValTok{1}\NormalTok{]] <-}\StringTok{ }\NormalTok{tabla_muestras}
\NormalTok{        lista[[}\DecValTok{2}\NormalTok{]] <-}\StringTok{ }\NormalTok{media_poblacional}
\NormalTok{        lista[[}\DecValTok{3}\NormalTok{]] <-}\StringTok{ }\NormalTok{varianza_poblacional}
\NormalTok{        lista[[}\DecValTok{4}\NormalTok{]] <-}\StringTok{ }\KeywordTok{data.frame}\NormalTok{(}\DataTypeTok{Muestra =} \KeywordTok{names}\NormalTok{(}\KeywordTok{colMeans}\NormalTok{(tabla_muestras)), }\DataTypeTok{mediamuestral =} \KeywordTok{unname}\NormalTok{(}\KeywordTok{colMeans}\NormalTok{(tabla_muestras))) }
\NormalTok{        lista[[}\DecValTok{5}\NormalTok{]] <-}\StringTok{ }\KeywordTok{data.frame}\NormalTok{(}\DataTypeTok{estimador =} \StringTok{"media_muestral"}\NormalTok{, }
                                 \DataTypeTok{Media =} \KeywordTok{mean}\NormalTok{(lista[[}\DecValTok{4}\NormalTok{]][, }\DecValTok{2}\NormalTok{]),}
                                 \DataTypeTok{Mediana =} \KeywordTok{median}\NormalTok{(lista[[}\DecValTok{4}\NormalTok{]][, }\DecValTok{2}\NormalTok{]),}
                                 \DataTypeTok{SD =} \KeywordTok{sd}\NormalTok{(lista[[}\DecValTok{4}\NormalTok{]][, }\DecValTok{2}\NormalTok{]), }
                                 \DataTypeTok{IQR =} \KeywordTok{IQR}\NormalTok{(lista[[}\DecValTok{4}\NormalTok{]][, }\DecValTok{2}\NormalTok{]), }
                                 \DataTypeTok{MAD =} \KeywordTok{mad}\NormalTok{(lista[[}\DecValTok{4}\NormalTok{]][, }\DecValTok{2}\NormalTok{]), }
                                 \DataTypeTok{Curtosis =}\NormalTok{ moments}\OperatorTok{::}\KeywordTok{kurtosis}\NormalTok{(lista[[}\DecValTok{4}\NormalTok{]][, }\DecValTok{2}\NormalTok{]), }
\NormalTok{                                 Asimetrí}\DataTypeTok{a =}\NormalTok{ e1071}\OperatorTok{::}\KeywordTok{skewness}\NormalTok{(lista[[}\DecValTok{4}\NormalTok{]][, }\DecValTok{2}\NormalTok{]))}
\NormalTok{        lista[[}\DecValTok{6}\NormalTok{]] <-}\StringTok{ }\KeywordTok{gather}\NormalTok{(}\KeywordTok{as.data.frame}\NormalTok{(lista[[}\DecValTok{6}\NormalTok{]])) }
        \KeywordTok{colnames}\NormalTok{(lista[[}\DecValTok{6}\NormalTok{]]) <-}\StringTok{ }\KeywordTok{c}\NormalTok{(}\StringTok{"Muestra"}\NormalTok{, }\StringTok{"Estadístico"}\NormalTok{)}
\NormalTok{        lista[[}\DecValTok{7}\NormalTok{]] <-}\StringTok{ }\KeywordTok{data.frame}\NormalTok{(}\DataTypeTok{estimador =} \StringTok{"estadístico"}\NormalTok{,}
                                 \DataTypeTok{Media =} \KeywordTok{mean}\NormalTok{(lista[[}\DecValTok{6}\NormalTok{]][, }\DecValTok{2}\NormalTok{]),}
                                 \DataTypeTok{Mediana =} \KeywordTok{median}\NormalTok{(lista[[}\DecValTok{6}\NormalTok{]][, }\DecValTok{2}\NormalTok{]),}
                                 \DataTypeTok{SD =} \KeywordTok{sd}\NormalTok{(lista[[}\DecValTok{6}\NormalTok{]][, }\DecValTok{2}\NormalTok{]), }
                                 \DataTypeTok{IQR =} \KeywordTok{IQR}\NormalTok{(lista[[}\DecValTok{6}\NormalTok{]][, }\DecValTok{2}\NormalTok{]), }
                                 \DataTypeTok{MAD =} \KeywordTok{mad}\NormalTok{(lista[[}\DecValTok{6}\NormalTok{]][, }\DecValTok{2}\NormalTok{]), }
                                 \DataTypeTok{Curtosis =}\NormalTok{ moments}\OperatorTok{::}\KeywordTok{kurtosis}\NormalTok{(lista[[}\DecValTok{6}\NormalTok{]][, }\DecValTok{2}\NormalTok{]), }
\NormalTok{                                 Asimetrí}\DataTypeTok{a =}\NormalTok{ e1071}\OperatorTok{::}\KeywordTok{skewness}\NormalTok{(lista[[}\DecValTok{6}\NormalTok{]][, }\DecValTok{2}\NormalTok{]))}
\NormalTok{        lista[[}\DecValTok{8}\NormalTok{]] <-}\StringTok{ }\KeywordTok{rbind}\NormalTok{(lista[[}\DecValTok{5}\NormalTok{]], lista[[}\DecValTok{7}\NormalTok{]])}
        \KeywordTok{names}\NormalTok{(lista) <-}\StringTok{ }\KeywordTok{c}\NormalTok{(}\StringTok{"tabla_muestras"}\NormalTok{, }\StringTok{"media_poblacional"}\NormalTok{, }\StringTok{"varianza_poblacional"}\NormalTok{, }\StringTok{"media_muestral"}\NormalTok{, }
                          \StringTok{"medidas_media_muestral"}\NormalTok{, }\StringTok{"estadistico"}\NormalTok{, }\StringTok{"medidas_estadistico"}\NormalTok{, }\StringTok{"tabla_comparacion"}\NormalTok{)}
\NormalTok{        lista}
\NormalTok{\}}



\NormalTok{resultados <-}\StringTok{ }\KeywordTok{muestreo}\NormalTok{(}\DataTypeTok{alpha =} \DecValTok{2}\NormalTok{, }\DataTypeTok{beta =} \DecValTok{2}\NormalTok{)}

\NormalTok{datosgrafico <-}\StringTok{ }\KeywordTok{inner_join}\NormalTok{(resultados}\OperatorTok{$}\NormalTok{media_muestral , resultados}\OperatorTok{$}\NormalTok{estadistico)}
\end{Highlighting}
\end{Shaded}

\begin{verbatim}
## Joining, by = "Muestra"
\end{verbatim}

\begin{Shaded}
\begin{Highlighting}[]
\KeywordTok{library}\NormalTok{(gt)}
\KeywordTok{library}\NormalTok{(tidyverse)}
\KeywordTok{gt}\NormalTok{(}\DataTypeTok{data =}\NormalTok{ resultados}\OperatorTok{$}\NormalTok{tabla_comparacion) }\OperatorTok\StringTok{ }\KeywordTok{tab_header}\NormalTok{(}\DataTypeTok{title =} \StringTok{"Hola em dic Marc"}\NormalTok{)}
\end{Highlighting}
\end{Shaded}

\captionsetup[table]{labelformat=empty,skip=1pt}
\begin{longtable}{lrrrrrrr}
\caption*{
\large Hola em dic Marc\\ 
\small \\ 
} \\ 
\toprule
estimador & Media & Mediana & SD & IQR & MAD & Curtosis & Asimetría \\ 
\midrule
media\_muestral & 0.5072706 & 0.5012344 & 0.07211760 & 0.08862303 & 0.06388053 & 3.077518 & 0.3099279 \\ 
estadístico & 0.5035930 & 0.5041133 & 0.08791211 & 0.13532182 & 0.09725723 & 2.656788 & 0.2748998 \\ 
\bottomrule
\end{longtable}

\begin{Shaded}
\begin{Highlighting}[]
\NormalTok{kableExtra}\OperatorTok{::}\KeywordTok{kable}\NormalTok{(resultados}\OperatorTok{$}\NormalTok{tabla_comparacion, }\DataTypeTok{format =} \StringTok{"latex"}\NormalTok{, }\DataTypeTok{booktabs =} \OtherTok{TRUE}\NormalTok{, }\DataTypeTok{longtable =} \OtherTok{TRUE}\NormalTok{) }
\end{Highlighting}
\end{Shaded}

\begin{longtable}{lrrrrrrr}
\toprule
estimador & Media & Mediana & SD & IQR & MAD & Curtosis & Asimetría\\
\midrule
media\_muestral & 0.5072706 & 0.5012344 & 0.0721176 & 0.0886230 & 0.0638805 & 3.077518 & 0.3099279\\
estadístico & 0.5035930 & 0.5041133 & 0.0879121 & 0.1353218 & 0.0972572 & 2.656788 & 0.2748998\\
\bottomrule
\end{longtable}

\hypertarget{r-markdown}{%
\subsection{R Markdown}\label{r-markdown}}

This is an R Markdown document. Markdown is a simple formatting syntax
for authoring HTML, PDF, and MS Word documents. For more details on
using R Markdown see \url{http://rmarkdown.rstudio.com}.

When you click the \textbf{Knit} button a document will be generated
that includes both content as well as the output of any embedded R code
chunks within the document. You can embed an R code chunk like this:

\begin{Shaded}
\begin{Highlighting}[]
\KeywordTok{summary}\NormalTok{(cars)}
\end{Highlighting}
\end{Shaded}

\begin{verbatim}
##      speed           dist       
##  Min.   : 4.0   Min.   :  2.00  
##  1st Qu.:12.0   1st Qu.: 26.00  
##  Median :15.0   Median : 36.00  
##  Mean   :15.4   Mean   : 42.98  
##  3rd Qu.:19.0   3rd Qu.: 56.00  
##  Max.   :25.0   Max.   :120.00
\end{verbatim}

\hypertarget{including-plots}{%
\subsection{Including Plots}\label{including-plots}}

You can also embed plots, for example:

\includegraphics{TRABAJO-BETA_files/figure-latex/pressure-1.pdf}

Note that the \texttt{echo\ =\ FALSE} parameter was added to the code
chunk to prevent printing of the R code that generated the plot.

\end{document}
